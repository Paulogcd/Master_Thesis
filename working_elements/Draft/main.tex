\documentclass{article}

\usepackage[utf8]{inputenc}
\usepackage{amsmath} 
\usepackage{amsfonts}
\usepackage{graphicx}
\usepackage{tabularx}
\usepackage{array}
\usepackage{booktabs}
\usepackage{listings}
\usepackage{hyperref}


% Bibliography:
\usepackage[style=authoryear]{biblatex}
\addbibresource{references.bib}
% To compile:
% pdflatex main.tex
% biber main
% pdflatex main.tex
% pdflatex main.tex

% Graphics: 
\graphicspath{ {./output/} }
\usepackage{float} % to force location of pictures, use "H"

% Author and Title:
\author{Paulo Gugelmo Cavalheiro Dias}
\title{Master Thesis: Climate change economic effect on health through temperature}

\begin{document}


\maketitle

\tableofcontents

\section{Introduction}

\subsection{Introduction to the subject}

Climate change is projected to strongly impact temperature distribution in the upcoming century. 
Therefore, it is important to try to identify the different effects of temperature on the economy.
One of the particular effect channels is health. 
How will temperature change affect the health status of the population, and therefore the economic production?

\subsection{Related literature}

In order to study this question, 

The relationship between temperature, health, and economics is deeply intricated 
in several literature fields. 

First, since the 2000s, the relationship on how climate and economics are intertwined has been studied in a more climate-related literature.

Second, health economics has identified important findings in the last years.

- Health economics: 
    - Health and productivity.
    - Health and life expectancy.
- Climate and economics: 
    - Hotter temperature and conflicts. 


\subsection{Research question and strategy}

The goal of this Master Thesis is to propose a simple approach to model the effect
of temperature on the health of individuals, and subsequently on the economy. 

In order to achieve this, two main parts are identified. 

First, an empirical part studying the link between temperature and health will be done. 

Second, a macroeconomic model is presented to explain the economic mechanisms that will be affected by the identified empirical relationships.

\section{Setting}

This section is dedicated to the presentation of the general setting
in which individuals live in this model.
First, the relationships between the different elements in a general framework will be presented.
Then, the data used will be presented and discussed. 
The methods used to estimate the functional forms of the relationships will then be 
presented, and will be refered as the specific case of this work.
Finally, results of the estimation process of these relationships will be presented and discussed.

\subsection{Goal}

\subsection{Formal Description}

At each period, an exogenous weather realization occurs, and individuals draw a health and living status. 

\subsubsection{History vectors}

In a general framework, we can think of the health of an individual at time $t$ as a vector $\mathcal{H}_t\in \mathbb{R}^{t}$ containing
all the health status of the individual throughout their life.
Similarily, we can think of the weather experienced by an individual at time $t$ as a vector $\mathcal{W}_t\in \mathbb{R}^{t}$ containing
all the weather conditions experienced by the same individual throughout their life.

\subsubsection{Health Status}

Let $H_{t}\in\Omega(H)$ be a random variable denoting the health status of an individual at time $t$.
The functional form of its distribution $f_{h}$ will depend on its sample space $\Omega(H)$. 
The past health history and the temperature also affect the probability distribution of health status.
Generically, we can therefore write: 

\begin{equation}
    H_{t}\sim f_{h}(\mathcal{H}_{t-1},\mathcal{W}_t)
\end{equation}

Also, it follows that: 

\begin{equation}
    \mathcal{H}_{t} = (H_{1},...,H_{t})
\end{equation}

\subsubsection{Living Status}

Let $L_{t}\in\{0,1\}$ be a random binary variable denoting the living status of an individual at time $t$.
It is determined by a Bernoulli distribution with parameter $p_t$ such that: 

\begin{equation}
    L_{t} \sim \mathcal{B}(p_{t})
\end{equation}

The probability parameter $p_{t}$ depends on their health, age, and temperature.
In a general approach, we can rewrite the first equation such as: 

\begin{equation}
    L_{t} \sim \mathcal{B}(p_{t}(\mathcal{H}_t,\mathcal{W}_t))
\end{equation}

As such, the probability of an individual to be alive at period $t$ is: 

\begin{equation}
    Pr(L_t = 1 | \mathcal{H}_t,\mathcal{W}_t ) = \prod_{j = 1}^{t} p_{j}(\mathcal{H}_j,\mathcal{W}_j)
\end{equation}

\subsection{Data}

Three main datasets were used to estimate these relationships: 
the Health and Retirement Study (HRS) dataset for
health status and survival,
the Berkeley Earth dataset,
and finally the Federal Reserve Bank of Saint Louis (FRED) dataset for 
other economic variables.

\subsubsection{HRS Data}

The HRS has a main survey performed every two years on a panel of individuals in the United States of America (USA). 
An exit survey occurs in parallel, that targets individuals identified as dead, in which questions are asked to relatives.
The exit survey was used to identify dead individual, for which the living status was noted as $0$ at the year of the survey. 
Dead individuals were kept in the final analyzed dataset if they had been observed in the immediate previous survey. 
To avoid mortality effects of the COVID-19, the latest year selected for this study was 2018.
Due to different encoding, the 2000 and prior surveys were not taken into account. 

Four variables were used in this dataset: Year of the survey, age of individuals, living status, and health status.
The age of individuals was determined based on the year of birth question, that was substracted to the year of the survey. 
Living status was encoded as $1$ if an individuals was present in the main survey, and $0$ if they were in the exit survey.
The health status of dead individuals was encoded as the same as in the previous survey. 
The health status was identified through the self-reported health.
This measure is less precise than alternative composites, based on weighted average of specific questions of health, 
but perform well enough and is more tractable.

In the HRS data, health status can take eight possible values: 


- 1: Excellent

- 2: Very Good

- 3: Good

- 4: Fair

- 5: Poor

- 8, -8, or 9: Non Available, Not answered, or refused to answer\footnote{This last category was removed from the final analyzed dataset.}\\





\begin{figure}[H]
    \includegraphics[width=\textwidth]{/Users/paulogcd/Documents/Master_Thesis/working_elements/Draft/output/histogram_1.png}
    \caption{Health Status distribution per Year, from the HRS data}
    
    
\end{figure}

\subsubsection{Climate Data}

The climate data of Berkeley Earth was used to collect information
on average annual or pluriannual global temperature on land.
The name of the chosen dataset was the "Land Monthly Average Temperature". 
% This file contains a detailed summary of the land-surface average 
% results produced by the Berkeley Averaging method.  Temperatures are 
% in Celsius and reported as anomalies relative to the Jan 1951-Dec 1980 
% average.  Uncertainties represent the 95% confidence interval for 
% statistical and spatial undersampling effects.
For each month, from 1880 to 2022, two anomaly extrema values are given. 
They correspond to the 95\% confidence interval for the
average temperature. 
The average temperature is computed as a deviation from the average annual 
temperature computed between Januray 1951 and December 1980. 

\begin{figure}[H]
    \includegraphics[width=\textwidth]{/Users/paulogcd/Documents/Master_Thesis/working_elements/Draft/output/figure_1.png}
    \caption{Average annual temperature from the Berkeley Dataset}
    
    The light blue area is delimited by the anomalies extrema of each year, and
    the dark blue line represents the average of the anomalies.
\end{figure}

\subsubsection{Economic Data}

Finally, the FRED data was used to get two economic variables. 
First, the annual Gross Domestic Product (GDP) of the USA. 
Second, the interest rate, whose average is then used to calibrate the economic model.

\subsection{Methods}

\subsubsection{Health Transition}

The HRS dataset provides rich information regarding the health status of individuals. 
To make use of the five different values of the self reported health, it was therefore decided not to
recode the variables into a binary health status variable $H\in\{Bad, Good\}$. 

Several methods were considered to estimate the relationship between current health and past health and other covariates. 
Given that health is here an discrete, ordinal variable, ordered response models were chosen.
More specifically, the ordered logit and ordered probit models were selected, due to their simplicity and broad usage \parencite{Wooldridge_2010}.

\subsubsection{Living Status}

The living status variable, being binary, did not represent a challenge
as great as the health transition estimation.

A simple logistic regression on age and health status yields findings similar with the
rest of the literature. 

\begin{figure}[H]
    \includegraphics[width=\textwidth]{/Users/paulogcd/Documents/Master_Thesis/working_elements/Draft/output/figure_2.png}
    \caption{Annual probability of survival as a function of age and health status}
\end{figure}


\subsection{Estimates}

\section{Model}

\subsection{Baseline specification }

The agents maximizes: 

$$ \max_{\{c_{t},l_{t},s_{t+1}\}_{t=1}^{100}}{\mathbb{E}\left[\sum_{t=1}^{100} \beta^{t}\cdot u(c_t,l_t)\right]}$$

Their utility function is: 

$$u(c_{t},l_{t}) = \frac{c_{t}^{1-\rho}}{1-\rho}-\xi_{t}\cdot \frac{l_{t}^{1+\varphi}}{1+\varphi}$$

With : 

-  $c_{t}$  the consumption

-  $l_{t}$  the quantity of labor supply provided by the agent

-  $h_{t}$  the health status

-  $w_{t}$  the weather variable, which is here temperature

-  $\xi_{t}$ the labor disutility coefficient

- $\rho$ the risk aversion coefficient

- $\varphi$ the Frisch elasticity
\\

The agent is subject to the following budget constraint:

$$c_{t} + s_{t+1} \leq l_{t}\cdot z_{t} + s_{t}\cdot(1+r_{t})$$

With: 

-  $c_t$ the consumption at period $t$

-  $s_{t+1}$ the savings for period $t+1$

-  $l_t$ the labor supply provided by the agent at period $t$

-  $z_t$ the productivity at time $t$

-  $s_{t}$ the savings available at the beginning of period $t$

-  $r_{t}$ the interest rate at period $t$

Also, agents are subject to the following borrowing constraint, defined as: 

$$s_{t+1}\geq \underline{s}, \forall t \in [\![1,T]\!]$$


We can note the First Order Conditions, such that: 

\begin{equation}
    c^{-\rho}_{t}\cdot z_{t} = \xi_{t}\cdot l_{t}^{\varphi} \iff
        \begin{cases}
        & c_t = \left[\frac{\xi_{t}\cdot l_{t}^{\varphi}}{z_{t}}\right]^{-\frac{1}{\rho}}\\ 
        & l_{t} = \left[\frac{c_{t}^{-\rho}z_{t}\cdot}{\xi_{t}}\right]^{\frac{1}{\rho}}
    \end{cases}
\end{equation}
And 
\begin{equation}
    c^{-\rho}_{t} = \beta \cdot \mathbb{E}\left[c^{-\rho}_{t+1}\cdot (1+r_{t+1})\right] + \gamma_{t}
\end{equation}

The first corresponds to the equalization of marginal benefit and
cost of labor, and the second corresponds to the Euler equation.

The first equilbrium condition implies an within decision,
driven by the labor disutility coefficient $\xi$ and the productivity $z$.
There is a unique mapping between consumption and labor at any period, to 
equalize the benefits and the costs of labor.

The second equilibrium condition implies an intertemporal decision.
The marginal utility of consumption at one period must be equal to the 
expected marginal utility of conusmption next period, discounted by 
the discounting factor $\beta$ and the interest rate next period $(1+r_{t+1})$, 
plus the marginal benefit of violating the borrowing constraint at the current period. 

It is now important to describe what the Expectation operator $\mathbb{E}$ entails here. 
In a generic formulation, one could expect the uncertainty to affect the interest rate at the next period, 
which is the reason $(1+r_{t+1})$ is within the operator.

Another specification could exclude any uncertainty from 
the interest rate. 
The uncertainty could then come from the health and survival draw. 

If the uncertainty only comes from these two draws, the expectation operator can be formalized such as: 

$$\mathbb{E}\left[c_{t+1}\right] \equiv p_{t+1}(\mathcal{H}_{t},\mathcal{W}_{t}) \cdot c_{t+1}$$

In this specification, the only uncertainty is whether the agent will be alive or not in next period.
The baseline model will first focus on this simplified assumption. 
Variations will be introduced later.

\subsection{Solution Impossibility}

\textbf{Proposition}
This maximization program is impossible to solve analytically in most cases\footnote{The proof of this proposition is in the Appendix.}.
\\

If we consider the model altogether, it is impossible to describe 
analytically the optimal policy functions of the three choice variables.
While the entire proof is available in the appendix, a quick explanation
is possible here.
First, the objective functions is linear with the savings at next period $s_{t+1}$, 
making it disappear from the F.O.C.s.
This term requires therefore the labor and consumption policies to be solved, 
and then plugged into the budget constraint, to have a solution. 
However, if we try to solve the two other policy functions, 
we end up with transcendantal equations of the form $a\cdot x^{\alpha} + b\cdot x + c = 0$, 
with $\alpha\notin \mathbb{N}$. 
This transcendantal equations can be overcome with specific combinations 
of parameters, among which $\varphi = -\rho$, in which case we obtain: 

$$EQUATION$$

To allow for more flexibility in the resolution, the model was mainly solved
analytically. The next section discusses the different methods used in order to do so.

\subsection{Numerical methods}

Several ways have been considered to solve this model numerically. 
This section is dedicated to the presentation of the different methods
used in order to do so. 
First, the auxiliary functions are presented. 
Second, the different main algorithms specifications and their performance are presented. 
Finally, the aggregation methods and different numerical results are discussed.

\subsection{Functions}

- Utility function, 
- Budget clearing function, 
- Bellman function, 
- Backwards function

\subsection{Algorithms}

This section details the different algorithms used to perform a numerical 
solution of the model. 
The general concept of each will be explained in words, and 
pseudo-code will illustrate them.

\subsubsection{Pure numerical value function iteration}

The pure numerical value function iteration algorithm consists
in verifying all possible 
combination of choice variables for each level of state variable 
to determine what is the best possible response given a certain
amount of state variable \footnote{The different steps of the algorithms and their source code are available online.
The steps and comments of the present work are available here: \url{https://www.paulogcd.com/Master_Thesis/},
and the documented replication package, coded in Julia, is available here: \url{https://www.paulogcd.com/Master_Thesis_Paulogcd_2025/}.}.

\begin{lstlisting}
for t from 100 to 1
    
    if t is equal to 100
        Bellman next period = zeros(length(s_range))
    end

    for s in s_range
        for c,l,s' in ranges
            
            # We check if this combination of choice variable 
            # respects the budget constraint. 
            bc = budget_clearing(c,l,s') 

            # If it does not,
            # we set the value function to a very low number.
            if bc < 0
                V[index_l,index_l,index_s',index_s] = -Inf 
            
            # If it does, we compute the value function 
            # for this combination of choice variables.
            else if bc >= 0
                V[index_c,index_l,index_sprime,index_s] =
                    utility(c,l,s') +
                    beta * p[t+1] * Bellman next period[s']
            end
        end
        Value_function[index_s,t] = max(V[index_s])
    end
    Bellman_next_period = Value_function[index_s,t]
end
\end{lstlisting}

Here, the algorithm goes through all the possible values of
$c$, $l$, and $s'$, without using any approximation obtained 
through the FOC mentioned above. 
This is quite computational-intensive, but 
has the advantage of not using analytical results, 
which can lead to approximation depending on the resolution of the 
ranges used.

\subsubsection{F.O.C. approximated value function iteration}

The FOC value function iteration algorithms make use of the two FOC
derived in the previous section. 
They are faster by order of magnitudes, but contain more errors.

Note that it is impossible to use the second FOC, i.e. the 
Euler equation, containing the 
Lagrangien multiplier $\gamma_{t}$ that
is unsolvable analytically. 

\subsubsection{Interpolated algorithms}

The interpolated algorithms use interpolation techniques to approximate 
the value of the next period Bellman equation. 
This interpolation can be implemented in the pure numerical 
algorithm, and in the FOC-approximated ones. 

They allow for a smoother shape of policy function. 
Upon running the interpolations,
we observe that 
heir results on speed performance is slightly negative, 
and their results on precision performance is ambiguous.

\subsubsection{Policy iteration algorithms}


\section{Results}



\section{Conclusion}

\section{References}

\printbibliography

\section{Appendix}


\subsection{Proof of Impossibility}

The associated Lagrangien is: 
\begin{equation}
    \begin{split}
        \mathcal{L}(c_{t},l_{t},s_{t+1};\lambda_t,\gamma_{t}) &
        = \mathbb{E}\Big[\sum_{t=1}^{100} \beta^{t}\cdot ((\frac{c_{t}^{1-\rho}}{1-\rho}-\xi_{t}\cdot\frac{l_{t}^{1+\varphi}}{1+\varphi}) \\
        & +\lambda_{t}\cdot \left(l_{t}\cdot z_{t}+s_{t}\cdot (1+r_{t})-c_{t}-s_{t+1}\right) \\ 
        & + \gamma_{t}\cdot \left(s_{t+1}-\underline{s}\right))\Big] \\ 
    \end{split}
\end{equation}

The First Order Conditions are the following:


$$\frac{\partial \mathcal{L}}{\partial c_{t}} = 0 \iff c_{t}^{-\rho} = \lambda_{t}$$


$$\frac{\partial \mathcal{L}}{\partial l_{t}} = 0 \iff \lambda_{t}\cdot z_{t} = \xi_{t}\cdot l_{t}^{\varphi}$$


$$\frac{\partial \mathcal{L}}{\partial s_{t+1}} = 0 \iff \lambda_{t} = \beta \cdot \mathbb{E}\left[\lambda_{t+1}\cdot (1+r_{t+1})\right] + \gamma_{t}$$

We first note that we must obtain a closed-form solution for $c_{t}$ and $l_{t}$ to obtain 
the optimal value of $s_{t+1}$. 

Replacing the expression of $\lambda_{t}$ in the two other equation yields: 

\begin{equation}
    c^{-\rho}_{t}\cdot z_{t} = \xi_{t}\cdot l_{t}^{\varphi} \iff
        \begin{cases}
        & c_t = \left[\frac{\xi_{t}\cdot l_{t}^{\varphi}}{z_{t}}\right]^{-\frac{1}{\rho}}\\ 
        & l_{t} = \left[\frac{c_{t}^{-\rho}z_{t}\cdot}{\xi_{t}}\right]^{\frac{1}{\rho}}
    \end{cases}
\end{equation}
And 
\begin{equation}
    c^{-\rho}_{t} = \beta \cdot \mathbb{E}\left[c^{-\rho}_{t+1}\cdot (1+r_{t+1})\right] + \gamma_{t}
\end{equation}

Assuming that the budget constraint binds, it becomes: 

$$c_{t} + s_{t+1} = l_{t}\cdot z_{t} + s_{t}\cdot(1+r_{t})
\iff 
c_{t} = l_{t}\cdot z_{t} + s_{t}\cdot(1+r_{t}) - s_{t+1} 
$$

Plugging this expression into the maximization program, 
it can therefore be rewritten such as: 

\begin{equation}
    \max_{\{l_{\text{t}},s_{t+1}\}} u(l_{\text{t}}) = \frac{\left[l_{t}\cdot z_{t} + s_{t}\cdot (1+r_{t}) - s_{t+1} \right]^{1-\rho}}{1-\rho}-
\xi_{t}\cdot\frac{l_{t}^{1+\varphi}}{1+\varphi}
\end{equation}

The F.O.C implies: 

\begin{equation}
    l_{t}^{\varphi}\cdot \xi_{t} = \left[l_{t}\cdot z_{t} + s_{t}\cdot(1+r_{t})- s_{t+1}\right]^{-\rho}\cdot z_{t}
\end{equation}

We can develop the decomposition of consumption if and only if $\rho \in \mathbb{N}$.
Indeed, this equation is of form $x = (x-\alpha)^{\beta} \cdot z$.
With $\beta\notin \mathbb{N}$, is a transcendantal equation.


If we try to solve it backwards, we now go to the last period. 
At the last period, $s_{t+1} = \underline{s}$ for sure:
Since there is no future, 
the agent will borrow as much as they can,
or will at least not save anything more than what is imposed 
by the constraint. 

For simplification, let $\underline{s}$ be fixed such that: $\underline{s} = 0$.
The new optimality condition is: 

\begin{equation}
    l_{t}^{\varphi}\cdot \xi_{t} = \left[l_{t}\cdot z_{t} + s_{t}\cdot(1+r_{t})\right]^{-\rho}\cdot z_{t}
\end{equation}

This is still a transcendantl equation, and the problem remain the same.

\textbf{Special cases:}

Now, if we plug the F.O.C. 2, (equation 7), in the budget constraint, we obtain: 

$$
\begin{cases}
    & c_t = \left[\frac{\xi_{t}\cdot l_{t}^{\varphi}}{z_{t}}\right]^{-\frac{1}{\rho}} \\
    & c_{t} = l_{t}\cdot z_{t} + s_{t}\cdot(1+r_{t}) - s_{t+1} 
\end{cases}
$$

$$\iff$$
$$ \left[\frac{\xi_{t}\cdot l_{t}^{\varphi}}{z_{t}}\right]^{-\frac{1}{\rho}} = l_{t}\cdot z_{t} + s_{t}\cdot(1+r_{t}) - s_{t+1} $$
$$\iff$$
$$ l_{t}^{-\frac{\varphi}{\rho}} \cdot \left(\frac{\xi_{t}}{z_{t}}\right)^{-\frac{1}{\rho}} = l_{t}\cdot z_{t} + s_{t}\cdot(1+r_{t}) - s_{t+1} $$
$$\iff$$
$$ l_{t}^{-\frac{\varphi}{\rho}} \cdot \left(\frac{\xi_{t}}{z_{t}}\right)^{-\frac{1}{\rho}} - l_{t}\cdot z_{t} - s_{t}\cdot(1+r_{t}) - s_{t+1} = 0 $$

This is a transcendantal equation of form 
$x^{\alpha}\cdot b - x\cdot y - a = 0$

Which admits a solution if and only if: $\frac{\varphi}{\rho} \in \mathbb{N}$

\end{document}